\subsection{Conclusions}
\begin{frame}\frametitle{Conclusions}
    \begin{block}{Problem}
		\centering{``How can the existing platform be extended in such a way that everyone in the FeedbackFruits community can contribute to the software development process?''}
	\end{block}
	
	\begin{enumerate}
	    \item Time + knowledge = extensions
	    \item `With FeedbackFruits I would like to \ldots'
	    \item Goals, challenges and subchallenges
	    \item Supporters, hype and process updates
	    \item Tool integrations
	\end{enumerate}
\end{frame}

% The ideal platform should make it possible to convert resources in the form of time and knowledge into an extension to the current FeedbackFruits platform. The goal of this extension should be to fulfill a particular need. A goal can be identified as the answer to finishing the phrase: "With FeedbackFruits I would like to ..". Each goal can be divided in challenges, sub-goals that need to be met in order to meet the goal. If these challenges are to large to be solved by one or two supporters, then the challenges must be divided into sub-challenges in such a way that they can be solved by one or two pioneers. This implies that the whole project can be solved by lots of different people by tackling smaller problems at the same time.

% To attract these supporters it should be possible to hype a goal via, for example social media. To keep supporters in the loop it should also be possible to post process updates that get relayed to all supporters.

% The wheel should not be reinvented. Therefore integrations should be used wherever possible. Two examples:
% 1. GitHub: Used by many programmers.
% 2. Chat client. It is possible to make one ourselves, but is better to use an existing tool that people are already familiar with.