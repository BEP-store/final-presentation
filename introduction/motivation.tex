\subsection{Subject \& motivation}
\begin{frame}\frametitle{Subject \& motivation}
    \begin{enumerate}
        \item User feedback
        \item Open-source developers
        \item Different toolsets
        \item Unification
    \end{enumerate}
\end{frame}
%1. Users have feedback but are mostly non-techinal,
%2. and developers lack domain knowledge.
% Open source is rising, knowledge gap between those two groups
%3. Users and developers use different toolsets.
%4. How to let these groups work together?

% Users of the clients platform are mostly non-technical people. The research report identifies GitHub as the most important tool for developers in the collaborative software development process. This tool focuses on developers and is not designed to be used by non-technical users. While the users would like to improve the clients platform, it appears to be difficult to translate their ideas to the developers because of a knowledge gap. The research conducted during this project shows that there are no existing implementations of a solution to this problem. The solution should bridge this gap in knowledge. Therefore, the contextual problem description can be defined as: ``How can the existing platform be extended in such a way that everyone in the FeedbackFruits community can contribute to the software development process?''

% 1. Technical knowledge is not always properly translated into function
% In addition to collecting feedback from their users, FeedbackFruits would like to boost community-engagement in improving education. It is the goal of the company to improve education for as many people as possible. The motivation behind stimulating community engagement is to create a community-driven ecosystem for innovation in education.