\subsection{Problem}

\begin{frame}\frametitle{Problem - Description}
    \begin{alertblock}{Problem}
		\centering{``How can the existing platform be extended in such a way that everyone in the FeedbackFruits community can contribute to the software development process?''}
	\end{alertblock}
	
\end{frame}

% The solution should bridge this gap in knowledge. Therefore, the contextual problem description can be defined as: “How can the existing platform be extended in such a way that everyone in the FeedbackFruits community can contribute to the software development process?”

\begin{frame}\frametitle{Problem - Analysis}
	\begin{enumerate}
        \item What can be learned from existing (non-software) communities?
        \item What current efforts exists to involve the community in the software development process?
        \item How are requirements established and used in software development?
        \item What limitations does the existing FeedbackFruits ecosystem impose on the software?
    \end{enumerate}
\end{frame}

% To anwser the four main research questions, the report will have four sections. The first research question zooms in on what can be learned from (non-software) communities. Before theories can be defined about what can be learned from all sorts of communities, a definition must be established of what a community is. Because communities exist in different sizes, with different structures and on different platforms, a few of these communities will be analysed in section 1.
% Some ways to involve the community in building software may be composed based on the previous section. However, this research will probably not be the only attempt to involve a software community in building software. The second research question is therefore focussed on existing platforms where software communities are involved in one way or another in the software development process. Section 2 will focus on the structure of the existing software communities. The techniques used by those communities and the way that the development process engages the community will also be examined.+

% Software development in its current form has some methodologies to establish requirements for a (big) software development project. The third research question will take a better look at the dynamics of requirements. These requirements are the basis of how the software will be built and are therefore an essential part of the software development process. In section 3, a few of the methods wich are used to establish requirements will be described. The dynamics of the requirements, like management and implementation, will also be given a closer look.
% The first three research question focus on determining the critical attributes of the problem in a general sense. However, because these attributes have to be applicable on the current FeedbackFruits-platform, that platform may pose some limitations. With research question four, the structure of the FeedbackFruits-platform will be examined, as well as the extensiblity of the platform. The result of this research question can be found in section 4.

% The most important aspects to this problem were identified in the research report as being: hype,requirements engineering, code and process updates (appendixD, chapter 5). Hype is necessary togain supporters that are willing to work on processing the feedback or feature requested by a user. Tobetter understand the how the feature should work, the feedback should be decomposed into smallchallenges via requirements engineering. These challenges should be converted to working code. Thecontributors should be informed of the current status of the feature through process updates. Thisprovides a feedback loop that ensures the challenges are tackled correctly.